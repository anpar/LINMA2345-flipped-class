



\begin{frame}{Folk Theorem}
    \textbf{Folk Theorems}
    \begin{itemize}
        \item Theorems that circulate among mathematicians but cannot be traced back to one individual
        \item Considered to have established status but generally no proof in complete form
        \item One notable one in Game Theory, that we are going to see today
    \end{itemize}
\end{frame}

\begin{frame}{Folk Theorem}
    \textbf{Enforceable payoff profile}
    \begin{itemize}
        \item Let $v_i= \min_{s_{-i}\in S_{-i}} \max_{s_{i}\in S_{i}} u_i(s_{-i},s_i)$.\\
        Payoff profile $r$: enforceable if $r_i\geq v_i$
    \end{itemize}
    \textbf{Feasible payoff profile}
    \begin{itemize}
        \item Payoff profile $r$: feasible if there exist rational, non-negative values $\alpha_a$ such that $\forall i$:\\
        $r_i=\sum_{a\in A} \alpha_a u_i(a)$, with $\sum_{a\in A} \alpha_a=1 $
    \end{itemize}
\end{frame}



\begin{frame}{Folk Theorem}
    \textbf{Formulation of the theorem:}\\
    \textit{Consider any $n$-player normal-form game $G$ and any payoff profile $r = (r_1,r_2,...,r_n)$. Then:}
    \begin{enumerate}
        \item \textit{If $r$ is the payoff profile for any Nash equilibrium s of the infinitely repeated $G$ with average rewards, then for each player $i$, $r_i$ is enforceable.}
        \item \textit{If $r$ is both feasible and enforceable, then $r$ is the payoff profile for some Nash equilibrium of the infinitely repeated $G$ with average rewards.}
    \end{enumerate}
    \textit{This version of the theorem is proven in the book "\textbf{Multiagent Systems}, Algorithmic, Game-Theoretic, and Logical Foundations"}
\end{frame}


\begin{frame}{Folk Theorem}
    \textbf{What does it mean?}\\
    \begin{itemize}
        \item The first part of the theorem means that we can enforce any Nash equilibrium of the infinitely-repeated game
        \item The second part states that any payoff profile that is feasible and enforceable is the payoff profile of a Nash equilibrium 
    \end{itemize}
\end{frame}

\begin{frame}{Folk Theorem}
    \textbf{Consequences}\\
    \begin{itemize}
        \item Interesting consequences: apparition of new Nash equilibrium's for certain games
        \item Example: prisoner's dilemna
        \begin{itemize}
            \item If game goes on long enough ($\delta \rightarrow 1$), cooperation of both players becomes a Nash equilibrium
        \end{itemize}
    \end{itemize}
\end{frame}



\begin{frame}{Folk Theorem}
    \begin{exampleblock}{Example}
        Consider the Prisoner's Dilemma in normal form.
        \begin{table}
            \begin{tabular}{c|cc}
                & {\color{red}c}    & {\color{red}d} \\
                \hline
                {\color{green}C}    & \payoff{-1}{-1}   & \payoff{-4}{~0} \\
                {\color{green}D}    & \payoff{~0}{-4}    & \payoff{-3}{-3} 
            \end{tabular}
            \caption{Prisoner's Dilemma in normal form.}
        \end{table}
    \end{exampleblock}
\end{frame}

\note{
    Simply prove that cooperation is feasible and enforceable:\\
    \begin{itemize}
        \item Enforceability
            $v_1=\min_{s_{-1}\in S_{-1}} \max_{s_{1}\in S_{1}} u_1(s_{-1},s_1) =-3$\\
            $v_2=\min_{s_{-2}\in S_{-2}} \max_{s_{2}\in S_{2}} u_2(s_{-2},s_2) =-3$\\
            $r_1=1\geq -3$,$r_2=1\geq -3 \rightarrow$ enforceable \\
        \item Feasibility
            $\alpha_{(c,C)}= 1$, rest $\alpha=0 \rightarrow$ feasible
    \end{itemize}
    By Folk theorem $\rightarrow$ Nash equilibrium
}



\begin{frame}{Take-home message \#2}
    \metroset{block=fill}
    \begin{block}{Take-home message \#2}
        Folk theorem:\\
        \begin{center}    
            Payoff is Nash $\leftrightarrow$ Payoff feasible and enforceable
        \end{center}
    \end{block}
\end{frame}
