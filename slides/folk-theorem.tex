\begin{frame}{Folk Theorem}
    \begin{block}{Folk Theorems}
        \begin{itemize}
            \item Theorems that circulate among mathematicians but cannot be traced back to one individual
            \item Considered to have established status but generally no proof in complete form
            \item One notable one in Game Theory, that we are going to see today
        \end{itemize}
    \end{block}
\end{frame}

\begin{frame}{Enforceability and feasability}
    \begin{itemize}
        \item Consider any $n$-player game $\Gamma = (N, D, u)$ and any payoff vector
        $u = (u_1, \dots, u_n)$.
        \item Let $v_i= \min_{d_{-i}\in D_{-i}} \max_{d_i\in D_i} u_i(d_{-i}, d_i)$ be
        the minmax value of player $i$.
    \end{itemize}
    \metroset{block=fill}

    \begin{block}{Enforceable payoff profile}
        A payoff profile $u$ is \textbf{enforceable} if $u_i \geq v_i$.
    \end{block}

    \begin{block}{Feasible payoff profile}
        A payoff profile $u$ is \textbf{feasible} if there exist rational, non-negative
        values $\alpha_d$ such that for all $i$, we can express $u_i$ as
        $\sum_{d\in D} \alpha_d u_i(d)$, with $\sum_{d\in D} \alpha_d = 1$.
    \end{block}
\end{frame}

\begin{frame}{Folk Theorem with average rewards}
    \metroset{block=fill}

    \begin{block}{The Folk theorem with average rewards}
        Consider any $n$-player normal-form game $\Gamma$ and any payoff profile
        $u = (u_1, \dots, u_n)$. Then\footnote{This version of the theorem is proven
        in the book "\textbf{Multiagent Systems}, Algorithmic, Game-Theoretic, and Logical
        Foundations".}
        \begin{enumerate}
            \item If $u$ is the payoff profile in any Nash equilibrium of the infinitely
            repeated $\Gamma$ with average rewards, then for each player $i$, $u_i$ is enforceable.
            \item If $u$ is both feasible and enforceable, then $u$ is the payoff profile for some
            Nash equilibrium of the infinitely repeated $\Gamma$ with average rewards.
        \end{enumerate}
    \end{block}
\end{frame}


\begin{frame}{Interpretation of the Folk theorem}
    \begin{exampleblock}{What does it mean?}
        \begin{itemize}
            \item The first part of the theorem means that we can enforce any Nash equilibrium
            of the infinitely-repeated game
            \item The second part states that any payoff profile that is feasible and enforceable
            is the payoff profile of a Nash equilibrium 
        \end{itemize}
    \end{exampleblock}
\end{frame}

\begin{frame}{Folk Theorem}
    \textbf{Consequences}\\
    \begin{itemize}
        \item Interesting consequences: apparition of new Nash equilibrium's for certain games
        \item Example: prisoner's dilemna
        \begin{itemize}
            \item If game goes on long enough ($\delta \rightarrow 1$), cooperation of both
            players becomes a Nash equilibrium
        \end{itemize}
    \end{itemize}
\end{frame}

\begin{frame}{Folk Theorem}
    \begin{exampleblock}{Example}
        Consider the Prisoner's Dilemma in normal form.
        \begin{table}
            \begin{tabular}{c|cc}
                & {\color{red}c}    & {\color{red}d} \\
                \hline
                {\color{green}C}    & \payoff{-1}{-1}   & \payoff{-4}{~0} \\
                {\color{green}D}    & \payoff{~0}{-4}    & \payoff{-3}{-3} 
            \end{tabular}
            \caption{Prisoner's Dilemma in normal form.}
        \end{table}
    \end{exampleblock}
\end{frame}

\note{
    Simply prove that cooperation is feasible and enforceable:\\
    \begin{itemize}
        \item Enforceability
            $v_1=\min_{s_{-1}\in S_{-1}} \max_{s_{1}\in S_{1}} u_1(s_{-1},s_1) =-3$\\
            $v_2=\min_{s_{-2}\in S_{-2}} \max_{s_{2}\in S_{2}} u_2(s_{-2},s_2) =-3$\\
            $r_1=1\geq -3$,$r_2=1\geq -3 \rightarrow$ enforceable \\
        \item Feasibility
            $\alpha_{(c,C)}= 1$, rest $\alpha=0 \rightarrow$ feasible
    \end{itemize}
    By Folk theorem $\rightarrow$ Nash equilibrium
}

\begin{frame}{Take-home message \#7}
    \metroset{block=fill}
    \begin{block}{Take-home message \#7}
        By the \textbf{Folk theorem} with average rewards
        \begin{center}    
            payoff Nash equilibrium $\leftrightarrow$ payoff feasible and enforceable.
        \end{center}
    \end{block}
\end{frame}
