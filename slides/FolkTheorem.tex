

\begin{frame}{Folk Theorem}
    \textbf{Folk Theorems}
    \begin{itemize}
        \item Theorems that circulate among mathematicians but cannot be traced back to one individual
        \item Considered to have established status but generally no proof in complete form
        \item One notable one in Game Theory, that we are going to see today
    \end{itemize}
\end{frame}


\begin{frame}{Folk Theorem}
    \textbf{Formulation of the theorem:}\\
    \textit{Consider any $n$-player normal-form game $G$ and any payoff profile $r = (r_1,r_2,...,r_n)$. Then:}
    \begin{enumerate}
        \item \textit{If $r$ is the payoff profile for any Nash equilibrium s of the infinitely repeated $G$ with average rewards, then for each player $i$, $r_i$ is enforceable.}
        \item \textit{If $r$ is both feasible and enforceable, then $r$ is the payoff profile for some Nash equilibrium of the infinitely repeated $G$ with average rewards.}
    \end{enumerate}
    \textit{This version of the theorem is proven in the book [MAS]}
\end{frame}


\begin{frame}{Folk Theorem}
    \textbf{What does it mean?}\\
    \begin{itemize}
        \item The first part of the theorem means that we can enforce any Nash equilibrium of the infinitely-repeated game
        \item The second part states that any payoff profile that is feasible and enforceable is the payoff profile of a Nash equilibrium 
    \end{itemize}
\end{frame}

\begin{frame}{Folk Theorem}
    \textbf{Consequences}\\
    \begin{itemize}
        \item Interesting consequences: apparition of new Nash equilibrium's for certain games
        \item Example: prisoner's dilemna
        \begin{itemize}
            \item If game goes on long enough ($\delta \rightarrow 1$), cooperation of both players becomes a Nash equilibrium
        \end{itemize}
    \end{itemize}
\end{frame}
