\begin{frame}{Infinitely Repeated Games and Utility}
    \begin{itemize}
        \item As we saw in the previous example, in the analysis of infinitely repeated games, we
        drop the assumption that all payoffs come at the end of the game. Why? \pause
        \item Player $i$ thus gets a potentially infinite sequence of payoffs
        \[ (u_i(1), u_i(2), u_i(3), \dots). \]
    \end{itemize}

    \begin{alertblock}{We need a criterion for \textit{ranking} payoff sequences!}
        \pause
        \begin{itemize}
            \item Sum of payoffs as in the finitely repeated case? \textbf{No}, may be
            infinite. \pause
            \item Other criteria exist: limit of average payoff, $\delta$-discounted average,
            {\color{gray}overtaking criterion}, etc.
        \end{itemize}
    \end{alertblock}
\end{frame}

\note{
    Answer to the ``Why?'': in the extensive form, we collect the payoff at the terminal nodes.
    But in an infinitely repeated game, the tree might be infinitely deep and we thus never
    receive the payoff!

    We want to turn this infinite sequence in a single number we can reason about!

    Sum of payoffs might actually be finite in some very particular condition: there is
    an absorbing state in which the player's payoffs are always 0 and the expected number
    of rounds until arrival at this absorbing state is always finite (see Myerson).

    The overtaking criterion is in gray because it will not be covered here.
}

\begin{frame}{Limit of average payoff}
    \metroset{block=fill}
    \begin{block}{Definition}
        The limit of average of a sequence of payoffs $(u_i(1), u_i(2), u_i(3), \dots)$ is
        \[ \lim_{K\to\infty} \frac{1}{K} \sum_{k=1}^K u_i(k). \]
    \end{block}
    
    \setbeamercolor{itemize item}{fg=orange,bg=white}
    \textbf{{\color{orange}Limitations}}\\
    \pause
    \begin{itemize}
        \item The limit may not exists. \pause
        \item Let's say you play a game every year: during the first 100 years, you lose \$1M
        each year. Afterward, you win \$1 each year.\\\
            
        What is the limit of average payoffs? \pause \textbf{{\color{orange}\$1}}...
    \end{itemize}
\end{frame}

\note{
    The first limitation is mathematical and can easily be identified by the audience: we should
    ask them first.
    We should maybe also mention than there exist a way to weaken the criterion so that
    the limit always exists (using limit-infimum and limit-supremum) and redirect to Myerson
    for more details.
}

\begin{frame}{$\delta$-discounted average payoff}
    \metroset{block=fill}
    \begin{block}{Definition}
        For any \textit{discount factor} $\delta$ such that $0 \le \delta < 1$, the $\delta$-\textit{discounted
        average} of a sequence of payoffs $(u_i(1), u_i(2), u_i(3), \dots)$ is
        \[ (1-\delta)\sum_{k=1}^{\infty} \delta^{k-1}u_i(k). \]
    \end{block}

    \setbeamercolor{itemize item}{fg=green,bg=white}
    \textbf{{\color{green}Interpretations}}\\
    \pause
    \begin{itemize}
        \item The game has a probability $\delta$ of continuing at each round, the $\delta$-discounted
        average in this case is simply the expected future payoff. \pause
        \item The discount factor $\delta$ represents a \textit{measure of patience}.
    \end{itemize}
\end{frame}

\note{
    \textbf{For any general repeated game, if the payoffs are bounded, then de $\delta$-discounted
    average of each player's sequence of payoffs is a finite number and is bounded in absolute
    value by the same number that bounds payoffs}.

    Before unraveling the item in the interpretation list, we might ask if they can provided
    an interpretation based on the previous example we just saw.

    Show quickly that if $\delta = 0$, the discounted average simply give $u_i(1)$.
    While in the limit as $\delta \to 1$, the weights of each $u_i(k)$ tend to be equal.
    Important to note that this is in the limit (as for $\delta = 1$ we just get 0...).
}

\begin{frame}{Take-home message \#3}
    \metroset{block=fill}
    \begin{block}{Take-home message \#3}
        There exists several critera for ranking sequence of payoffs $(u_i(1), u_i(2), u_i(3), \dots)$
        \begin{itemize}
            \item \textit{limit of average payoffs}
            \[ \lim_{K\to\infty} \frac{1}{K} \sum_{k=1}^K u_i(k). \]
            {\color{orange}Might not exist, counter-intuitive in some case}.
            \item $\delta$\textit{-discounted average payoff}
            \[ (1-\delta)\sum_{k=1}^{\infty} \delta^{k-1}u_i(k). \]
            {\color{green}Always bounded if the payoffs are bounded, easy interpretations}. 
        \end{itemize}
    \end{block}
\end{frame}
