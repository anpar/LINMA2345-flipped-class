\begin{frame}{Let's again play the repeated Prisoner's Dilemma}
    \begin{exampleblock}{Example}
        Consider again the Prisoner's Dilemma in normal form.
        \begin{table}
            \begin{tabular}{c|cc}
                & {\color{red}c}    & {\color{red}d} \\
                \hline
                {\color{green}C}    & \payoff{-1}{-1}   & \payoff{-4}{~0} \\
                {\color{green}D}    & \payoff{~0}{-4}    & \payoff{-3}{-3} 
            \end{tabular}
            \caption{Prisoner's Dilemma in normal form.}
        \end{table}
    
        We will roll a (fair) dice after each round
        \begin{itemize}
            \item if the result is 1, the game stops,
            \item otherwise the game continues.
        \end{itemize}
        At the end of each round, you collect your payoff (and thus observe what
        the other has played).
    \end{exampleblock}
\end{frame}

\note{
    Draw a table on the blackboard and collect the actions/outcomes of each pair
    of students after each round.

    At the end, quickly compare the result. Ask students to motivate their strategy
    choice (especially if it differs from what a rational player would do).

    In theory, student should more cooperate than in the previous case. However,
    because people already tend to cooperate in the finitely repeated case, the
    difference might not be significant on the small sample we have.
}

\begin{frame}{Can a cooperative strategy be a Nash Equilibrium?}
    \begin{block}{Cooperative strategy}
        \textit{cooperate at each round, until the other defect,
        then defect forever} and assume both players play accordingly.
    \end{block}

    \begin{exampleblock}{Is it a Nash Equilibrium?}
        Neither player can expect to gain by any unilateral deviation at any point
        in time (i.e., for any history of past moves).
        
        Two possible history of past moves in this case
        \begin{enumerate}
            \pause
            \item \textbf{one of the player has defected in the past}: both players are then always defecting
            and neither player could gain by deviating alone to cooperation, \pause
            \item \textbf{neither player has ever defected in the past}: both players are then always
            cooperating.
            Could it be more profitable for one player to defect? Less obvious, see blackboard!
        \end{enumerate}
    \end{exampleblock}
\end{frame}
    
\note{
    The total number of rounds follows a geometric distribution with expected value 6. That is, the
    probability that we play \textit{exactly} $k$ rounds is given by
    \[ p(k) = \left(\frac{5}{6}\right)^{k-1}\frac{1}{6}. \]

    The following formula is useful in the following reasoning
    \[ \sum_{k=1}^{\infty} kz^k = \frac{z}{(1-z)^2}. \]
}

\note{
    If both players follows the strategy and cooperate, they will each get an expected total future
    payoff of
    \begin{align*}
        \sum_{k=1}^{\infty} (-1k)\cdot p(k) &= -\frac{1}{6} \sum_{k=1}^{\infty} k\left(\frac{5}{6}\right)^{k-1}\\
                                            &= -\frac{1}{5} \sum_{k=1}^{\infty} k\left(\frac{5}{6}\right)^k \\
                                            &= -\frac{1}{5} \frac{5/6}{(1-5/6)^2} \\
                                            &= -\frac{1}{5} \frac{5}{6}\cdot 36 \\
                                            &= -6.
    \end{align*}
}

\note{
    If one of the players deviate and decide to defect, he will get 0 first and then -4 forever because the
    other player will always defect to punish him, the deviating player will then get an expected total future
    payoff of
    \begin{align*}
        0 + \sum_{k=2}^{\infty} (-3k)\cdot p(k) &= -\frac{1}{2} \sum_{k=2}^{\infty} k\left(\frac{5}{6}\right)^{k-1}\\
                                                &= -\frac{1}{2} \frac{6}{5} \sum_{k=2}^{\infty} k\left(\frac{5}{6}\right)^k \\
                                                &= -\frac{3}{5} \left(-\frac{5}{6} + \sum_{k=1}^{\infty} k\left(\frac{5}{6}\right)^k \right) \\
                                                &= -\frac{3}{5} \left(-\frac{5}{6} + 30 \right) \\
                                                &= -\frac{3}{5} \frac{175}{6} = -17.5 < -6.
    \end{align*}
}

\begin{frame}{Interpretation of the cooperative Nash Equilibrium}
    \begin{exampleblock}{How can we explain this cooperative equilibrium?}
        \begin{itemize}
            \item At each round, both players believe that there is a high probability that they
            will play again. The hope of \textbf{inducing a future cooperative behavior by the other
            player} can give each player an incentive to be generous.
            \item In addition, both players know that \textbf{defecting is} forever \textbf{punished} by
            the other player so that defecting, while more profitable in the short-term, is not profitable
            in the long-term.
        \end{itemize}
    \end{exampleblock}
\end{frame}

\begin{frame}{Take-home message \#2}
    \metroset{block=fill}
    \begin{block}{Take-home-message \#2}
        Rational behavior in a repeated games with a potentially infinite time horizon may be
        very different from rational behavior in the corresponding game played once (or a finite
        and known number of times).
    \end{block}
\end{frame}
